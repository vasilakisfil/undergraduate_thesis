

\chapter{Επίλογος και Ανοιχτά Προβλήματα}\label{ch:conclusion}
Σε αυτό το κεφάλαιο εκτθέτονται τα αποτελέσματα από αυτή την διπλωματική καθώς επίσης και μελλοντικές ερευνιτικές προκλήσεις που προκύπτουν από αυτή την εργασία.
\section{Ανασκόπηση και Συμπεράσματα}
Με την ανάπτυξη της τεχνολογίας των ασύρματων δικτύων αισθητήρων και των εφαρμογών τους όπως είναι η παρακολούθηση του φυσικού περιβάλλοντος, υπάρχει αυξημένο
ενδιαφέρον τα συστήματα αυτά να διατηρούν την λειτουργική τους κατάσταση για αρκετά χρόνια. Όμως το κύριο πρόβλημα σε αυτά τα συστήματα είναι η ανικανότητά τους
να αποθηκεύσουν μεγάλη ποσότητα ενέργειας προκειμένου να επιτευχθεί ο στόχος της αέναης λειτουργίας τους. Το πρόβλημα αυτό μπορεί να ξεπεραστεί με την εκμετάλευση
μιας πολύ καινούργιας τεχνολογίας, την τεχνολογία της ασύρματης μετάδοσης ενέργειας. Αυτή η διπλωματική ασχολείται με την αποδοτική αξιοποίηση και ενσωμάτωση της
τεχνολογίας της ασύρματης μετάδοσης ενέργειας στα ασύρματα δίκτυα αισθητήρων και γενικώς με την αποδοτική διαχείρηση της ενέργειας σε τέτοια συστήματα.

Αρχικά γίνεται μια εισαγωγή και παρουσιάζονται τα γενικά χαρακτηριστικά των ασύρματων δικτύων αισθητήρων (κεφάλαιο \ref{ch:intro-wsns}). Αμέσως μετά παρουσιάζονται οι
πιο γνωστές τεχνικές μείωσης της κατανάλωσης ενέργειας (κεφάλαιο \ref{ch:energy_reduction}) και επεξηγείται η απόδοσή τους. Γίνεται αναφορά σε αποδοτικά
πρωτόκολλα δρομολόγησης καθώς και τα διάφορα είδη τους όπως π.χ. ιεραρχικά ή εξισορρόπησης ενέργειας πρωτόκολλα. Επίσης επεξηγούνται οι δυνατότητες των πρωτοκόλλων
δρομολόγησης που χρησιμοποιούν κινητούς κόμβους και κινητές Πηγές καθώς και σε συστήματα που χρησιμοποιούν εναλλακτικούς τρόπους εξασφάλισης ενέργειας όπως είναι η
απορρόφηση της ηλιακής ενέργειας. Στην συνέχεια (κεφάλαιο \ref{ch:wrsns}) γίνεται μια ιστορική αναδρομή στην τεχνολογία της ασύρματης μετάδοσης ενέργειας. Η
τεχνολογία επεξηγείται πλήρως ενώ παρουσιάζονται και οι ιδιότητές της. Αμέσως μετά (κεφάλαιο \ref{ch:strategies_solution}) ορίζεται το πρόβλημα που πρέπει να λυθεί
ενσωματώνοντας αυτή την τεχνολογία στα ασύρματα δίκτυα αισθητήρων. Αποδεικνύεται οτι το πρόβλημα είναι NP-πλήρης (NP-complete) και δίνεται ένα άνω φράγμα του
προβλήματος βασισμένο σε γραμμικό προγραμματισμό. Στην συνέχεια προτείνονται ευρετικές στρατηγικές και αλγόριθμοι οι οποίοι δίνουν αποδοτική λύση στο πρόβλημα. Τέλος
διεξάγονται εξονυχιστικά πειράματα των στρατηγικών και αλγορίθμων που προτάθηκαν τα οποία παρέχουν σημαντικές πληροφορίες για τις ιδιότητες της τεχνολογίας αυτής
καθώς και για την αλληλεπίδραση που έχει με τα ασύρματα δίκτυα αισθητήρων.

Από αυτή την εργασία προέκυψαν τα εξής συμπεράσματα:
\begin{itemize}
\item Η τεχνολογία της ασύρματης μετάδοσης ενέργειας θα πρέπει να ενσωματωθεί στα ασύρματα δίκτυα αισθητήρων καθώς όπως προέκυψε (\ref{sc:result1}) η διοχεύτεση
μέρος της συνολικής ενέργειας του δικτύου στον κινητό φορτιστή μπορεί να αυξήσει δραματικά τον χρόνο ζωής του δικτύου. Επομένως μπορεί να λυθεί αποδοτικά το πρόβλημα
της περιορισμένης ενέργειας στα ασύρματα δίκτυα αισθητήρων χρησιμοποιώντας τέτοιους αυτόματους μηχανισμούς οι οποίοι θα μπορούν να φορτίζονται από μια βάση στην άκρη
του δικτύου και να φορτίζουν τους κόμβους του δικτύου στο οποίο είναι αδύνατη η ανθρώπινη προσπέλαση.
\item Η τεχνολογία της ασύρματης μετάδοσης ενέργειας μπορεί να προκαλέσει εξισορρόπησης ενέργειας στο δίκτυο (εικόνες \ref{fig:1exp_4_1}, \ref{fig:1exp_4_2},
\ref{fig:1exp_4_3}) αφού η ενέργεια δίνεται μόνον στους κόμβους που πραγματικά χρειάζονται την ενέργεια.
\item Η μερική φόρτιση έχει πολύ καλύτερα αποτελέσματα ειδικά όταν χρησιμοποιούνται αλγόριθμοι δρομολόγησης οι οποίοι εξασφαλίζουν μια μερική ή ολική εξισορρόπησης
ενέργειας (\ref{sc:result2}).
\item Το βέλτιστο ποσοστό ενέργειας που μπορεί να δοθεί στον κινητό φορτιστή (\ref{sc:result3}) είναι κοντά στο 20\% για πρωτόκολλα μερικής ή ολικής εξισορρόπησης
ενέργειας ή το 50\%-60\% για δενδρικής μορφής πρωτόκολλα όπως το Directed Diffusion \cite{directed_diffusion} και το Greedy\cite{greedy_protocol}.
\item H βέλτιστη τροχιά για τον κινητό φορτιστή είναι η προσαρμοστική τροχιά καθώς προσαρμόζεται ανάλογα με τις συνθήκες του δικτύου\ref{sc:result4}.
\end{itemize}

\section{Μελλοντικές Αναζητήσεις}
Η έρευνα στα ασύρματα επαναφορτιζόμενα δίκτυα αισθητήρων ουσιαστικά μόλις έχει ξεκινήσει και επομένως υπάρχουν πολλές κατευθύνσεις για μελλοντική έρευνα.

Η ανάλυση και η απόδειξη αποδοτικών προσεγγιστικών αλγορίθμων με εγγυήσεις απόδοσης αποτελεί μια ερευνητική πρόκληση. Θα πρέπει να σημειωθεί οτι η αναγωγή στο
θεώρημα \ref{th:np-complete} για την απόδειξη της δύσκολης πολυπλοκότητας του προβλήματος δεν διαφυλάσει τους προσεγγιστικούς λόγους (π.χ. γνωτοί αποδοτικοί
αλγόριθμοι για το γεωμετρικό πρόβλημα του περιοδεύοντος πωλητή μπορεί να μην είναι και τόσο αποδοτικοί για πρόβλημα της δρομολόγησης του κινητού φορτιστή, CDDP).
Επίσης, υπάρχει ενδιαφέρον να μελετηθούν και να αναλυθούν στρατηγικές και τροχιές φόρτισης για μη ομοιόμορφες κατανομές ανάπτυξης των κόμβων του δικτύου. Διαφορετικά
μοντέλα ως προς την ανάπτυξη των κόμβων μπορεί να οδηγήσουν σε τελείως διαφορετικούς αλγορίθμους για τον κινητοό φορτιστή. Μια άλλη ερευνητική κατεύθυνση είναι
η μοντελοποίηση, η ανάλυση και η εξομοίωση άλλων μοντέλων φόρτισης όπως είναι στο \cite{multiple_devices} το οποίο επιτρέπει την φόρτιση πολλαπλών συσκευών
ταυτόχρονα. Επίσης η ανάλυση του λόγου επιτυχίας ως μια επιπλέον μετρική θα μπορούσε να ερευνηθεί αν και τα αποτελέσματα της αύξησης του χρόνου ζωής του δικτύου
σηματοδοτούν και τον καλό λόγο επιτυχίας.


Όπως δείχθηκε από αυτή την εργασία, η διοχεύετευση ενός μέρους της ενέργειας του δικτύου στον κινητό φορτιστή (και συγκεκριμένα το 20\%) συμφέρει περισσότερο καθώς
αυξάνει δραματικά τον χρόνο ζωής του δικτύου. Όμως σε αυτή την εργασία δεν έχει ληφθεί υπόψη η ενέργεια που σπαταλάει ο φορτιστής για την κίνησή του. Επίσης, ο
φορτιστής σε αυτή την εργασία εκτελεί εργασίας μόνο φόρτισης, δηλαδή δεν συλλέγει δεδομένα από τους κόμβους. Μια πολύ ενδοαφέρουσα ερευνητική κατεύθυνση είναι να
εξεταστεί αν συμφέρει περισσότερο ο κινητός φορτιστής να εκτελεί μόνο χρέη φόρτισης ή να συλλέγει ταυτόχρονα και δεδομένα από τους κόμβους, λαμβάνοντας υπόψην όμως
την ενέργεια που σπαταλάται για την κίνηση, την ενέργεια διαθέσιμη για φόρτιση και την ενέργεια διαθέσιμη για την επικοινωνία με τους κόμβους ως μια \textbf{ενιαία}
ενέργεια που έχει διαθέσιμη ο κινητός φορτιστής. Θα πρέπει να αναλυθεί σε ποιο σημείο θα πρέπει να δoθεί περισσότερο βάρος, ανάμεσα δηλαδή στην φόρτιση και την
συλλογή δεδομένων, με κύριο στόχο την επέκταση του χρόνου ζωής του δικτύου αλλά και άλλες μετρικές όπως ο λόγος επιτυχίας και η χρονοκαθυστέρηση (latency). Η φόρτιση
των κόμβων είναι ακριβή για τον κινητό φορτιστή αλλά σημαντική για την λειτουργία του δικτύου ενώ οι συλλογή δεδομένων είναι πολύ πιο φθηνή (ανάλογα, φυσικά, με τον
όγκο των δεδομένων) αλλά σημαντική για τα δεδομένα του δικτύου και επομένως η λύση δεν είναι προφανής.
