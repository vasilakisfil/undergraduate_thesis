
\begin{titlingpage}
\begin{center}

% Upper part of the page
\includegraphics[width=0.25\textwidth]{upatras_logo.jpg}\\[1cm]

\HUGE 	ΠΑΝΕΠΙΣΤΗΜΙΟ ΠΑΤΡΩΝ\\
\LARGE	Πολυτεχνική Σχολή\\
\Huge	Τμήμα Μηχανικών Ηλεκτρονικών Υπολογιστών και Πληροφορικής\\[1.5cm]

\LARGE Διπλωματική Εργασία\\[0.1cm]


% Title
\HRule \\[0.4cm]
{ \HUGE \bfseries Αποδοτική Διαχείρηση Ενέργειας σε Ασύρματα Επαναφορτιζόμενα Δίκτυα Αισθητήρων}\\[0.4cm]

\HRule \\[0.5cm]

% Author and supervisor
\begin{minipage}{0.4\textwidth}
\begin{flushleft} \large
\emph{Συγgραφέας:}\\
Φίλιππος Βασιλάκης, \\AM: 3895\\ vasilakis@ceid.upatras.gr
\end{flushleft}
\end{minipage}
\begin{minipage}{0.4\textwidth}
\begin{flushright} \large
\emph{Επιβλέπων:} \\
Σωτήρης Νικολετσέας, \\Επίκουρος Καθηγητής\\ nikole@cti.gr
\end{flushright}
\end{minipage}

\vfill

% Bottom of the page
{\large Μάιος 2012}

\end{center}

\end{titlingpage}

% \frontmatter
\tableofcontents
\chapter{Πρόλογος}
Τα ∆ίκτυα Αισθητήρων αποτελούνται από ένα μεγάλο αριθμό μικρών αυτόνομων συσκευών, που αλληλεπιδρούν με το άμεσο περιβάλλον τους μέσω αισθητήρων, επικοινωνούν μεταξύ
τους ασύρματα και συνεργάζονται φέροντας εις πέρας εργασίες που δε θα μπορούσε να ολοκληρώσει μία μόνο συσκευή. Κάθε συσκευή του δικτύου διαθέτει περιορισμένη
υπολογιστική δύναμη και ενεργειακούς πόρους, επομένως η όσο το δυνατόν λιγότερη κατανάλωση ενέργειας είναι βασικό πρόβλημα των δικτύων αισθητήρων για τη
μεγιστοποίησητου χρόνου Σωής τους. Τα πρώτα χρόνια η έρευνα επικεντρώθηκε στην διατύπωση των θεμελιωδών ιδιοτήτων που χαρακτηρίζουν ένα Ασύρματο ∆ίκτυο Αισθητήρων.
Για αυτό τον λόγο μελετήθηκαν απλοϊκά σενάρια που περιελάμβαναν ως επί το πλείστον στατικά δίκτυα με ισχυρές υποθέσεις, για παράδειγμα ως προς την ακτίνα εμβέλειας ή
την διάταξη των αισθητήρων.  ́Ενα τέτοιο σενάριο μελέτης
συνήθως περιελάμβανε ένα πλήθος κόμβων αισθητήρων σε μία περιοχή ενδια-
φέροντος, επιφορτισμένους με την συλλογή πληροφοριών από το περιβάλλον.
Στην συνέχεια η πληροφορία που ανιχνεύτηκε προωθείται προς το στατικό
κέντρο ελέγχου χρησιμοποιώντας κατανεμημένα πρωτόκολλα δρομολόγησης
και πολυβηματικές μεταδόσεις.
Αργότερα, οι ερευνητές άρχισαν να μελετούν όλο και πιο σύνθετα δίκτυα
αισθητήρων εισάγοντας στα μοντέλα αραιές τοπολογίες δικτύων, εμπόδια και
κίνηση. Με αυτό τον τρόπο ήλθαν αντιμέτωποι με νέες προκλήσεις, πολλές
από τις οποίες δεν έχουν γίνει ακόμη κατανοητές σε βάθος. Η εργασία που
κρατάτε στα χέρια σας είναι από τις πρώτες σε διεθνές επίπεδο που αντιμε-
τωπίζει τις προκλήσεις Ασύρματων ∆ικτύων Αισθητήρων που χαρακτηρίζονται
από έντονη και όχι προβλεπόμενη κίνηση των αισθητήρων. Χρησιμοποιών-
τας μοντέλα που βασίζονται σε μαρκοβιανές αλυσίδες για την ακριβή προ-
σομοίωση της κίνησης, προτείνεται ένα καινούργιο πρωτόκολλο γρήγορης
και αποδοτικής συλλογής δεδομένων από κινητά κέντρα ελέγχου. Επίσης,
προτείνονται και μια σειρά από επεκτάσεις του ίδιου πρωτοκόλλου που λαμ-
βάνουν υπόψη τους ευρύτερη γκάμα τοπολογιών και βελτιώνουν περαιτέρω
τις μετρικές.
Τα προτεινόμενα πρωτόκολλα αξιολογήθηκαν πειραματικά μέσω διεξο-
δικής προσομοίωσης, χρησιμοποιώντας ποικίλες τιμές για βασικές παραμέ-
τρους του δικτύου και συγκρίθηκαν με υπάρχουσες ευρέως αποδεκτές με-
θόδους. Τα αποτελέσματα δείχνουν ότι τόσο ο χρόνος παράδοσης των μηνυ-
μάτων, όσο και η ενέργεια που απαιτείται διατηρούνται σε χαμηλά επίπεδα,
βελτιώνοντας σημαντικά την προηγούμενη σχετική έρευνα.
\chapter{Ευχαριστίες}
\begin{abstract}
Θα ήθελα να ευχαριστήσω τον κ. Σωτήρη Νικολετσέα, Αναπληρωτή Καθηγητή
του Πανεπιστημίου Πατρών, για την επίβλεψη της παρούσας εργασίας, τη
καθοδήγηση και την εμπιστοσύνη που επέδειξε στο πρόσωπό μου.
Επίσης, θα ήθελα να ευχαριστήσω τους εξαιρετικούς δάσκαλους και κα-
θηγητές μου, που από τα παιδικά μου χρόνια σφυρηλάτησαν τον χαρακτήρα
μου.
Τέλος, θέλω να ευχαριστήσω θερμά την οικογένειά μου για την στήριξη
που πάντα μου προσφέρει, καθώς και τους συνοδοιπόρους μου σε αυτό το
ταξίδι.

\end{abstract}

% \mainmatter
