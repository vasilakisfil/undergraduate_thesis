
%%%%%%%%%%%%%%%%%%%%%%%%%%%%%%%%%%%%%%%%%%%%%%% titlepage %%%%%%%%%%%%%%%%%%%%%%%%%%%%%%%%%%%%%%%%%%%%%%%%%%%%%%%
\begin{titlingpage}
\begin{center}

% Upper part of the page
\includegraphics[width=0.25\textwidth]{images/upatras_logo.jpg}\\[1cm]

\HUGE 	ΠΑΝΕΠΙΣΤΗΜΙΟ ΠΑΤΡΩΝ\\
\LARGE	Πολυτεχνική Σχολή\\
\Huge	Τμήμα Μηχανικών Ηλεκτρονικών Υπολογιστών και Πληροφορικής\\[1.5cm]

\LARGE Προπτυχιακή Διπλωματική Εργασία\\[0.1cm]


% Title
\HRule \\[0.4cm]
{ \HUGE \bfseries Αποδοτική Διαχείρηση Ενέργειας σε Ασύρματα Επαναφορτιζόμενα Δίκτυα Αισθητήρων}\\[0.4cm]

\HRule \\[0.5cm]

% Author and supervisor
\begin{minipage}{0.4\textwidth}
\begin{flushleft} \large
\emph{Συγgραφέας:}\\
Φίλιππος Βασιλάκης, \\AM: 3895\\ vasilakis@ceid.upatras.gr
\end{flushleft}
\end{minipage}
\begin{minipage}{0.4\textwidth}
\begin{flushright} \large
\emph{Επιβλέπων:} \\
Σωτήρης Νικολετσέας, \\Επίκουρος Καθηγητής\\ nikole@cti.gr
\end{flushright}
\end{minipage}

\vfill

% Bottom of the page
{\large Μάιος 2012}

\end{center}

\end{titlingpage}
%%%%%%%%%%%%%%%%%%%%%%%%%%%%%%%%%%%%%%%%%%%%%%% titlepage ended %%%%%%%%%%%%%%%%%%%%%%%%%%%%%%%%%%%%%%%%%%%%%%%%%%%%%%%

% \frontmatter
\chapter{Πρόλογος}
Ένα Ασύρματο Δίκτυο Αισθητήρων(ΑΔΑ) αποτελείται από πολλές διασπαρμένες μικρές, αυτόνομες, ασύρματα συνδεδεμένες συσκευές. Κάθε συσκευή λαμβάνει μετρήσεις από το
άμεσο περιβάλλον της όπως θερμοκρασία, υγρασία, κλπ ενώ ταυτόχρονα οι πόροι όπως η μνήμη και η ενέργεια που έχει στη διάθεσή της η κάθε συσκευή είναι πολύ
περιορισμένοι.
Η ιδέα αναπτύχθηκε στα τέλη της δεκαετίας 1990 και τα πρώτα πειραματικά δίκτυα κατασκευάστικαν στις αρχές του 2000. Σε αυτά τα χρόνια η έρευνα επικεντρώθηκε κυρίως
στην διατύπωση των θεμελιωδών ιδιοτήτων που χαρακτηρίζουν ένα Ασύρματο Δίκτυο Αισθητήτων. Από τους πρώτους ερευνητές που άρχισαν να ασχολούνται με αυτά τα
δίκτυα υπήρχε η κοινή διαπίστωση οτι οι διαφορές από τα κλασσικά Ασύρματα Δίκτυα(ad-hoc networks) ήταν πολύ μεγάλες κυρίως στους στόχους του κάθε δικτύου.
Διατυπώθηκαν τα πρώτα προβλήματα που έπρεπε να λυθούν σε ένα τέτοιο δίκτυο τα οποία κυρίως αφορούσαν την εξοικονόμηση ενέργειας.

Σχεδόν μία δεκαετία μετά, τα Ασύρματα Δίκτυα Αισθητήρων ακόμα προκαλούν τους επιστήμονες στο θέμα της ελαχιστοποίησης της κατανάλωσης της ενέργειας. Αποδοτικοί
αλγόριθμοι, καινοτόμες τοπολογίες, ριζοσπαστικές ιδέες που ξεφεύγουν από τα κλασσικά μοντέλα, όπως κίνηση κάποιων αισθητήρων, τελευταίως και η εκμετάλευση
εναλλακτικών μορφών ενέργειας έχουν χρησιμοποιηθεί προκειμένου να μειώσουν την κατανάλωση ενέργειας στο δίκτυο. Το ανολοκλήρωτο όνειρο της επιστημονικής κοινότητας
είναι να γίνει πραγματικότητα ένα ΑΔΑ στο οποίο όλες οι συσκευές να έχουν μικρή παρόμοια κατανάλωση ενέργειας και χρόνο ζωής το άπειρο.

Τελευταία, μία καινούργια ανακάλυψη έρχεται να αναταράξει τα ΑΔΑ: Η Ασύρματη μεταφορά ενέργειας.

\chapter{Ευχαριστίες}
\begin{abstract}
Θα ήθελα να ευχαριστήσω τον κ. Σωτήρη Νικολετσέα, Επίκουρο Καθηγητή του Πανεπιστημίου Πατρών, για την επίβλεψη της παρούσας εργασίας, τη καθοδήγηση και την
εμπιστοσύνη που επέδειξε στο πρόσωπό μου όταν του ζήτησα μία καινοτόμα διπλωματική. Η εμπίστοσύνη αυτή σίγουρα με έκανε να δουλέψω ακόμα σκληρότερα. Θα ήθελα
να ευχαριστήσω τον Διδακτορικό Μάριο Αγγελακόπουλο ο οποίος με την πολυετή εμπειρία και τις σοφές παρατηρήσεις του συνέβαλε σημαντικά στα παρακάτω αποτελέσματα.
Επίσης ευχαριστώ τον συνάδελφο, απόφοιτο, Φάνη Ράπτη που βοήθησε σημαντικά να βγουν έγκαιρα τα αποτελέσματα. Είμαι σίγουρος οτι έχει λαμπρό μέλλον ως μέλος
της υπόλοιπης ομάδας.

Πάνω από όλους όμως θα ήθελα να ευχαριστήσω τους γονείς μου για τις συμβουλές τους και την στήριξη και που μου παρείχαν καθόλη την
διάρκεια των σπουδών μου. Χωρίς αυτούς τίποτα από όλα αυτά δεν θα είχε πραγματοποιηθεί.

\end{abstract}
\newpage
\tableofcontents
% \mainmatter
