
%%%%%%%%%%%%%%%%%%%%%%%%%%%%%%%%%%%%%%%%%%%%%%% titlepage %%%%%%%%%%%%%%%%%%%%%%%%%%%%%%%%%%%%%%%%%%%%%%%%%%%%%%%
\begin{titlingpage}
\begin{center}

% Upper part of the page
\includegraphics[width=0.25\textwidth]{images/upatras_logo.jpg}\\[1cm]

\HUGE 	ΠΑΝΕΠΙΣΤΗΜΙΟ ΠΑΤΡΩΝ\\
\LARGE	Πολυτεχνική Σχολή\\
\Huge	Τμήμα Μηχανικών Ηλεκτρονικών Υπολογιστών και Πληροφορικής\\[1.5cm]

\LARGE Προπτυχιακή Διπλωματική Εργασία\\[0.1cm]


% Title
\HRule \\[0.4cm]
{ \HUGE \bfseries Αποδοτική Διαχείρηση Ενέργειας σε Ασύρματα Επαναφορτιζόμενα Δίκτυα Αισθητήρων}\\[0.4cm]

\HRule \\[0.5cm]

% Author and supervisor
\begin{minipage}{0.4\textwidth}
\begin{flushleft} \large
\textit{Συγγραφέας:}\\
Φίλιππος Βασιλάκης, \\AM: 3895\\ \textit{vasilakis@ceid.upatras.gr}
\end{flushleft}
\end{minipage}
\begin{minipage}{0.4\textwidth}
\begin{flushright} \large
\textit{Επιβλέπων:} \\
Σωτήρης Νικολετσέας, \\Επίκουρος Καθηγητής\\ \textit{nikole@cti.gr}
\end{flushright}
\end{minipage}

\vfill

% Bottom of the page
{\large Ιούλιος 2012}

\end{center}

\end{titlingpage}
\begin{titlingpage}
\begin{center}

% Upper part of the page
\includegraphics[width=0.25\textwidth]{images/upatras_logo.jpg}\\[1cm]

\HUGE 	UNIVERSITY OF PATRAS\\
\LARGE	School of Engineering\\
\Huge	Computer Engineering and Informatics Department\\[1.5cm]

\LARGE Undergraduate Diplome Thesis\\[0.1cm]


% Title
\HRule \\[0.4cm]
{ \HUGE \bfseries Efficient Energy Management in Wireless Rechargeable Sensor Networks}\\[0.4cm]

\HRule \\[0.5cm]

% Author and supervisor
\begin{minipage}{0.4\textwidth}
\begin{flushleft} \large
\textit{Συγγραφέας:}\\
Filippos Vasilakis, \\RN: 3895\\ \textit{vasilakis@ceid.upatras.gr}
\end{flushleft}
\end{minipage}
\begin{minipage}{0.4\textwidth}
\begin{flushright} \large
\textit{Επιβλέπων:} \\
Sotiris Nikoletseas, \\Associate Professor\\ \textit{nikole@cti.gr}
\end{flushright}
\end{minipage}

\vfill

% Bottom of the page
{\large July 2012}

\end{center}

\end{titlingpage}
%%%%%%%%%%%%%%%%%%%%%%%%%%%%%%%%%%%%%%%%%%%%%%% titlepage ended %%%%%%%%%%%%%%%%%%%%%%%%%%%%%%%%%%%%%%%%%%%%%%%%%%%%%%%

\frontmatter
\chapter{Declaration}
The contents of this thesis are the results of original research and have not been submitted for a high degree to any university or institution. Much of the work in
this thesis has been published in the following conference paper:
\begin{itemize}
\item Constantinos Marios Angelopoulos, Sotiris E. Nikoletseas, Theofanis P. Raptis, Christoforos Raptopoulos, \textbf{Filippos Vasilakis}, \textit{Efficient Energy
Management in Wireless Rechargeable Sensor Networks}, in the 15th ACM International Conference on Modeling, Analysis and Simulation of Wireless and Mobile
Systems 2012 (ACM MSWiM ’12), Paphos, Cyprus.
\end{itemize}
\chapter{Ευχαριστίες}
\begin{abstract}
Θα ήθελα να ευχαριστήσω τον κ. Σωτήρη Νικολετσέα, Επίκουρο Καθηγητή του Πανεπιστημίου Πατρών, για την επίβλεψη της παρούσας εργασίας, τη καθοδήγηση και την
εμπιστοσύνη που επέδειξε στο πρόσωπό μου όταν του ζήτησα μία καινοτόμα διπλωματική. Η εμπίστοσύνη αυτή σίγουρα με έκανε να δουλέψω ακόμα σκληρότερα.

Επίσης θα ήθελα να ευχαριστήσω την υπόλοιπη ομάδα: αρχικά τον έμπειρο διδακτορικό Μάριο Αγγελακόπουλο ο οποίος συνέχεια έδινε σαφής καθοδηγήσεις και σοφές
παρατηρήσεις όπου συνέβαλε πολύτιμα στην εργασία αυτή˙ τον συνάδελφο, απόφοιτο, χιουμορίστα Φάνη Ράπτη που βοήθησε σημαντικά να βγουν έγκαιρα τα
αποτελέσματα˙ τον μεταδιδακτορικό Χριστόφορο Ραπτόπουλο που με το αστείρευτο ταλέντο του έδωσε άμεσα λύσεις στα προβλήματα που προκύψαν.

Πάνω από όλους όμως θα ήθελα να ευχαριστήσω τους γονείς μου για τις συμβουλές τους και την στήριξη και που μου παρείχαν καθόλη την
διάρκεια των σπουδών μου. Χωρίς αυτούς τίποτα από όλα αυτά δεν θα είχε πραγματοποιηθεί.

\end{abstract}
\vspace{-10pt}
\chapter{Abstract}
\begin{abstract}
\vspace{-30pt}
Almost one decade later, wireless sensor networks (WSNs) still challenge scientists to solve the problem of minimizing energy consumption. Efficient routing
algorithms, innovate topologies, radical ideas that go beyond classical models, such as mobile sensors, and most recently the exploitation of alternative forms of
energy have been proposed in order to reduce energy consumption in the network. The incomplete dream of the scientific community is to realize a wireless sensor
network in which all devices have very small, balanced, energy consumption and lifetime close to infinity.

Simultaneously, at the end of 2007 scientists from university of MIT, presented a paper which proved analytically that it is possible to achieve wireless energy
transfer over long distances with efficacy close to 40\%. In 2008 they presented a real system that was able to transfer power wirelessly in a distance of 2 meters
and efficacy of 45\% while later other researchers improved the performance even more. Thus, it seems that the dream of perpetual operation of a wireless
sensor network can become true by embedding this technology through a mobile node which acts as a charger which charges the static nodes.

In this work, firstly, every possible known technique is presented which can help to minimize the energy consumption in a WSN. Then, a formal definition is provided
for the charging problem, which is proved to be computational hard and an upper bound is given using linear programming. In order to better understand the
impact that charger's integration into WSNs has, it is assumed that the total initial available energy to the network, $E_{total}$, is $E_{total}  = E_{sensors} +
E_{MC}^{init}$ where $E_{sensors}$ is the static sensor's initial energy and $E_{MC}^{init}$ is the energy that is given initially to the charger.

Under this model and using 3 completely different routing protocols, the following are investigated:
\begin{itemize}
\item Is it better to equip the mobile charger with a part of the total initial energy, $E_{total}$, in order to charge the sensors?
\item Which charging strategy (partial vs full) maximizes network's lifetime?
\item What percentage of the total initial energy, $E_{total}$, best maximizes network's lifetime?
\end{itemize}
Last but not least, numerous trajectories of mobile charger which use only local information are studied and compared to a heuristic global knowledge charger in
respect to network's lifetime, connectivity, coverage and energy balance.

\end{abstract}
\vspace{-10pt}
\chapter{Επιτομή}
\begin{abstract}
\vspace{-30pt}
Σχεδόν μία δεκαετία μετά, τα Ασύρματα Δίκτυα Αισθητήρων (ΑΔΑ) ακόμα προκαλούν τους επιστήμονες στο θέμα της ελαχιστοποίησης της κατανάλωσης της ενέργειας. Αποδοτικοί
αλγόριθμοι, καινοτόμες τοπολογίες, ριζοσπαστικές ιδέες που ξεφεύγουν από τα κλασσικά μοντέλα, όπως κίνηση κάποιων αισθητήρων, ενώ τελευταίως και η εκμετάλευση
εναλλακτικών μορφών ενέργειας έχουν χρησιμοποιηθεί προκειμένου να μειώσουν την κατανάλωση ενέργειας στο δίκτυο. Το ανολοκλήρωτο όνειρο της επιστημονικής κοινότητας
είναι να γίνει πραγματικότητα ένα ασύρματο δίκτυο αισθητήρων στο οποίο όλες οι συσκευές να έχουν πολύ μικρή, παρόμοια, κατανάλωση ενέργειας και χρόνο ζωής κοντά στο
άπειρο.

Ταυτόχρονα, στα τέλη του 2007 επιστήμονες από το πανεπιστήμιο του ΜΙΤ, παρουσίασαν μια εργασία στην οποία αποδείξαν αναλυτικά οτι είναι δυνατό να επιτευχθεί ασύρματη
μεταφορά ενέργειας σε μεγάλη απόσταση και με μεγάλη απόδοση, κοντά στο 40\%. Το 2008 παρουσίασαν πραγματικό σύστημα που μετέφερε ασύρματα ενέργεια σε απόσταση 2
μέτρων με απόδοση 45\% ενώ αργότερα άλλοι ερευνητές βελτίωσαν την απόδοση ακόμα περισσότερο. Φαίνεται λοιπόν οτι το όνειρο της αέναης λειτουργίας ενός ασύρματου
δικτύου αισθητήρων, μπορεί να πραγματοποιηθεί αξιοποιώντας αυτή την τεχνολογία μέσω ενός κινητού κόμβου ο οποίος λειτουργεί ως φορτιστής και φορτίζει τους στατικούς
κόμβους του δικτύου.

Στην εργασία αυτή, αρχικά παρουσιάζονται όλες οι τεχνικές που μπορούν να εφαρμοστούν σε ένα ΑΔΑ προκειμένου να υπάρξει μείωση της κατανάλωσης της ενέργειας. Στην
συνέχεια, ορίζεται αυστηρώς το πρόβλημα του κινητού φορτιστή, αποδεικνύεται οτι είναι NP-πλήρης (NP-complete) και δίνεται ένα άνω φράγμα της λύσης σε γραμμικό
προγραμματισμό. Για να υπάρχει καλύτερη σύγκριση στην επίδρασης που έχει η ενσωμάτωση του κινητού φορτιστή στα συστήματα ΑΔΑ, θεωρείται οτι η συνολική αρχική
διαθέσιμη ενέργεια του δικτύου, $E_{total}$, είναι $E_{total}  = E_{sensors} + E_{MC}^{init}$ όπου $E_{sensors}$ είναι η αρχική ενέργεια που δίνεται στους στατικούς
κόμβους και $E_{MC}^{init}$ η αρχική ενέργεια που δίνεται στον κινητό φορτιστή.

Κάτω από αυτό το μοντέλο και χρησιμοποιώντας 3 τελείως διαφορετικά πρωτόκολλα δρομολόγησης, ερευνώνται πειραματικά τα εξής:
\begin{itemize}
\item Συμφέρει να διοχετευθεί μέρος της συνολικής ενέργειας, $E_{total}$, στον φορτιστή ο οποίος να φορτίζει τους κόμβους;
\item Ποια στρατηγική φόρτισης (μερική ή ολική) βελτιστοποιεί τον χρόνο ζωής του δικτύου;
\item Τι ποσοστό της συνολικής ενέργειας, $E_{total}$, πρέπει να δοθεί στον φορτιστή για να βελτιστοποιηθεί ο χρόνος ζωής του δικτύου;
\end{itemize}

Τέλος, ερευνώνται διάφορες τροχιές που μπορεί να κάνει ο κινητός φορτιστής χρησιμοποιώντας μόνο τοπική γνώση και συγκρίνονται με μια ευρετική τροχιά καθολικής γνώσης
ως προς την περαιτέρω αύξηση του χρόνου ζωής του δικτύου, της συνδεσιμότητας, της κάλυψης και την εξισορρόπιση ενέργειας μέσα στο δίκτυο.

\end{abstract}
\newpage
\tableofcontents
% \mainmatter
